\chapter{Conclusion}
\vspace{-0.1cm}
The goal of the thesis was to design and implement the migration tool for the data management plans created in the Data Stewardship Wizard.

The thesis starts with the research of the state-of-the-art of the \gls{dsw} application.
After the research, the reader is acquainted with the analysis of the solution along with its requirements.
Based on defined requirements, individual use cases and their scenarios are discussed.

In the design chapter, the application design together with its architecture and primary classes are discussed.

Then, the development process, used technologies, and final application are introduced.

In the last chapter, the testing case is demonstrated describing how the application functionality was validated.

The outputs are two application modules, which may be integrated into the existing system.
The first module is a server application written in the Haskell programming language.
The second module is a web application frontend for the first module, written in the Elm language.

Working with such technologies was a huge personal challenge because not only I did not write a production-ready application in those technologies before, I did not even write a frontend or backend applications up until now.
Another unknown field for me was Data Stewardship itself.
I had to acquaint myself with two books \cite{b1-os, b2-ds} to fully understand the domain of the problem.

\vspace{-0.3cm}
\section*{Project future}
\vspace{-0.1cm}
Shortly, both modules should become part of the production application available to the general public.
The system itself is in active development; this means all major changes should be reflected in those modules too.
Therefore, the output will probably stay in active development.
In my opinion, it is the actual usability of the implemented application what makes this thesis exceptional.
