\chapter{Introduction}

Už od nepaměti věda sloužila k rozšiřování kolektivního vědomí společnosti.
Byla však výsadou odborníků a výsledky jejich výzkumů nebyly přístupné široké veřejnosti.
S rozmachem technologií jednadvacátého století se ale i toto mění.

Věda už neslouží pouze úzké skupině výzkumníků, ale nachází uplatnění takřka v každém oboru, se kterým se běžně setkáváme.
Strategická rozhodnutí v komerčním sektoru jsou dnes téměř výhradně založena na základě analýzi nasbíraných dat.
Zapojení veřejnosti ve zpracování výsledků výzkumů je tak nevyhnutelné.

Zpřístupnění vědy není ale důležité pouze pro laickou veřejnost, slouží i dalším výzkumníkům, kteří na výsledky exitujících experimentů chtějí navazovat.
Problematikou přístupnosti a znovupoužitelnosti výzkumů se zabývá hnutí Open Science.

Investice do vědy a výzkumu z peněz daňových poplatníků se jen v České Republice vyšplhají v roce 2019 na necelých 36 miliard korun\cite{rvvi-budget}.
V Evropské unii se pak navíc v rámci projektu Horizon Europe bude v letech 2021 až 2027 do vědy investovat více než 2,5 bilionu korun\cite{euc-horizon-budget}.

Je to právě odpovědnost k veřejným prostředkům, kvůli které by takto financovaný výzkum měl být přístupný (lidsky, ale i strojově), volně dostupný a na jeho výsledky by mělo být možné navázat.
Formát publikací pomocí vědeckých článků, který fungoval po celé dvacáté století, už dnes není dostačující.
Výsledky často nejsou zpracovatelné strojově nebo otevřeně (chráněné pomocí platebních bran),
naměřená data nejsou srozumitelná (opět člověku nebo stroji) a závěry výzkumu nejsou reprodukovatelné.

Hnutí Open Science si klade za cíl takový přístup k vědě změnit.
Zastřešuje doporučení pro výzkumníky -- způsoby jakými sdílet data, spolupracovat, šířit své a využívají již existující výkumy a dělat je tak znovypoužitelnými.
Základním požadavkem Open Science proto je, aby výzkumná data a všechny související nástroje splňovaly principy \glsentryshort{fair}.

Tyto principy uvádějí, že výzkumná data by měla splňovat čtyři základní vlastnosti: \gls{fair}.
Data by tedy měla být nalezitelná (pro člověka i stroj), přístupná (otevřeně, pomocí standardních protokolů), univerzálně zpracovatelná (se standardním názvoslovím, bez použití proprietárních formátů) a znovupoužitelná (strojově, s adekvátní licencí a dobře popsaná).

Protože se jedná pouze o principy, jejich implementace není striktně daná.
Jednou z možných implementací je \gls{dsw}.
Jedná se o systém vyvíjený organizací ELIXIR\footnote{Elixir desc} (konkrétně buňkami ELIXIR CZ a ELIXIR NL).

\section*{Cíl práce}