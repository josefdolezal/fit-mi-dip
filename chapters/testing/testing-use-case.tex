\section{Testing use case}\label{sec:testing-use-case}

In this section, I will demonstrate a non-trivial use case proving correctnes of the solution.
Firstly, I will specify an input state of the testing case and acquaint the reader with expected result for such input.
Then, I will summarize migrator steps as described in the application design in chapter \ref{cptr:design} and compare the actual result with the expected one.

\subsection{Test input state}

For the input, I will assume a list of events, which are needed to be presented to the user during the migration process.
The given list will be visualized in a tree structure representing original knowledge model on which the events are applied (in oppose to questionnaire structure, it ignores user's replies).
In this tree, I will use following notation for each node: \inbox{bgcode}{\texttt{[STATE] (ID) Title $<$Type$>$}}, where:

\begin{itemize}
    \item \texttt{[STATE]} stands for difference state (either \texttt{D} for deleted, \texttt{U} for unchanged, \texttt{A} for added or \texttt{M} for modified),
    \item \texttt{(ID)} stands for node's unique identifier,
    \item \texttt{Title} stands for node's display name,
    \item \texttt{$<$Type$>$} only applies to a question node and represent its type. 
\end{itemize}

To successfully demonstrate correctnes of the migrator solution, the events must cover non-trivial customizations.
Therefore, all possible difference states are taken into account.
The upgrade contains following events:

\begin{enumerate}
    \item \texttt{EditKnowledgeModelEvent} with identifier \texttt{KM},
    \item \texttt{EditQuestionEvent} with identifer \texttt{Q2},
    \item \texttt{AddAnswerEvent} with identifier \texttt{A2},
    \item \texttt{DeleteChapterEvent} with identifier \texttt{C3}.
\end{enumerate}

There are two important events.
One is adding answer to template item question type and the other is deleting chapter.

By adding an answer to a item templates question type, the application needs to take into account user's replies and create the change overview structures accordingly.
To fully test this behaviour, the input state assumes that there is exactly one user's reply (filled item template) to this question.

The chapter deletion event demonstrates how application should behave while deleting node composed from other nodes -- in this case, the chapter is composed from nested questions.

\begin{figure}[H]
	\dirtree{%
		.1 [M] (KM) Common ELIXIR Knowledge Model.
        .2 [U] (C1) Chapter 1.
        .3 [U] (Q1) Question 1.1 <Value>.
        .3 [M] (Q2) Question 1.2 <Item template>.
        .4 [U] (Q3) Question A <Value>.
        .4 [U] (Q4) Question B <Options>.
        .5 [U] (A1) Answer 1.
        .5 [C] (A2) Answer 2.
        .2 [U] (C2) Chapter 2.
        .3 [U] (Q5) Question 2.1 <Value>.
        .2 [D] (C3) Chapter 3.
        .3 [U] (Q6) Question 3.1 <Value>.
        .3 [U] (Q7) Question 3.2 <Value>.
    }
    \caption{Visualization of knowledge model customization events}\label{fig:cust-events}
\end{figure}

The figure \ref{fig:cust-events} shows how events in context of existing knowledge model for better understanding.

\subsection{Expected output}

The expected output contains five changes for the list of four customization events.
The summary of mapping customization events to changes visible to the user follows.

\subsection*{EditKnowledgeModelEvent}

This event should be presented as a single change.
Because it is an modification event, the application presents knowledge model title and description in the original and new versions.

The change is previewed in both panels, each displaying appropriate version of the knowledge model.

\subsection*{EditQuestionEvent}

In this case, editing question should be shown as a single change.
The change should inform the user that there are two replies.

The change is previewed in both panels, each displaying appropriate version of the question.

\subsection*{AddAnswerEvent}

Adding an answer to a question included in item template should be shown as change as many times as the user replied to the question.
In this case, user replied by filling two item templates and therefore the added question should be shown twice -- each time in context of one of the answers.

The change is displayed only in the right panel.

\subsection*{DeleteChapterEvent}

To ensure simplicity of the migration tool, the user should not be bothered by informations about deletion of nested.
Because of that, the chapter deletion should be shown as a single change and nested questions should not be shown.

The change is displayed only in the left panel.
