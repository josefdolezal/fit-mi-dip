\section{Implementation}\label{sec:implementation}

In this section, I describe in detail how features designed in chapter \ref{cptr:design} are actually implemented.
Sections are appropriately divided into parts according to the application where it was implemented.

\subsection{Initialize questionnaire migration}

The initialization part of the migration process is handled on both client and server side.
In the following sections, I will describe each part individually.

\subsubsection*{Server-side implementation}

On the server side, the initialization is an one step process.
The server expose and \gls{rest} \gls{api} route accessible on \route{POST /questionnaires/:qtnUuid/migrations} (described in \ref{sec:server-interface}).

The route is registered in shared application router using its \gls{url}, \gls{http} method and handling function.
The registration is shown in code example \ref{code:route-registration}.

\hscode{code:route-registration}{Registration of the \gls{rest} route}{route-registration.hs}

The initialization itself is simple, as it only requires to validate input data and store the migration state.
The state is a uncomplicated data structure which is composed from the migrated questionnaire (still based on the older version of the knowledge model) and identifier of the knowledge model.

Once the request pass validations, its data are passed are passed to \texttt{QuestionnaireMigrationService} module.
This module is resposinble for migration consistency -- it always transitions from one consisten state to another.

After the questionnaire is created, the stored data are enriched by compiled knowledge models and difference events and returned to the client.

By creating separate structure for keeping migration state, it is secured that the original version will never be modified during migration.
This approach is more complex than doing the state modification in-place (by modifying the original questionnaire directly), but significantly helps to keep system consistency.

\subsubsection*{Client-side implementation}

On the client side, this process requires two steps.
In the first step, the application needs to fetch informations about available knowledge models which user can migrate to.
This action is similar to upgrading knwoledge models itself and the whole code for such functionality was reused from existing code base (programmed by Ing. Jan Slifka\cite{mt-slifka}).

Once the data are fetched, the application will filter out invalid data (older versions of the knowledge model) and displays modal window with available options (shown in wireframe \ref{fig:wf1}).
Once the user selects version for migration the application sends data to the server.
After server responds, the application will load the migration detail and display it to the user.

The code used for transition between application states is shown in example \ref{code:create-migration}.

\elmcode{code:create-migration}{Transitioning between states during migration initialization}{create-migration.elm}

\subsection{Cancel questionnaire migration}

Now I would like acquaint the reader with implementatiton of the migration cancelling feature.

\subsubsection*{Server-side implementation}

On the server side, the cancellation is done by deleting existing questionnaire state.
By deleting initialized state, the original questionnaire will appear in exactly the same state as it was before the migration was initialized.

The endpoint is available at \route{DELETE /questionnaires/:qtnUuid/migrations} route.

Once the user sends request to the server, the it is validated by handler layer.
After validation, the data are processed using service which takes care of converting request data into internal representation.
The deletion itself is done using \texttt{DAO} as shown in example \ref{code:delete-dao}.

\hscode{code:delete-dao}{DAO module handling migration state deletion}{delete-dao.hs}

The result of the database action is converted into \gls{http} status code and returned to the client application with empty body.

\subsubsection*{Client-side implementation}

The client application offers user to cancel migration only in case, where the server sent questionnaire data with appropriate flag (as shown in wireframe \ref{fig:wf5}).
When the user choose to cancel the migration, the application sends request to the server.
Once the server reposponds to the request, list of questionnaires is reloaded to correspond with latest state available at server.

\subsection{Create migration context}

\subsubsection*{Server-side implementation}

\subsubsection*{Client-side implementation}

\subsection{Update migrated questionnaire state}

\subsubsection*{Server-side implementation}

\subsubsection*{Client-side implementation}

\subsection{Finalize migration}

Finilizing the migration is the last step to upgrade the questionnaire knowledge model version.

\subsubsection*{Server-side implementation}

Once the client requests to finalize the migration, server needs to do following tasks:

\begin{enumerate}
    \item create new questionnaire with migrated answers,
    \item delete the migration state.
\end{enumerate}

The first step ensures, that new questionnaire is created alongside the original one.
Together with questionnaire, replies and question flags are copied.

The second step reverts the original questionnaire into original state (can be migrated again).
Once the migration is finished, user's list of questionnaire will contain the new questionnaire alongside the orginal.

The \gls{api} endpoint is available at \route{PUT /questionnaires/:qtnUuid/migrations} and returns empty body with appropriate \gls{http} status code.

\subsubsection*{Client-side implementation}

The client application use similar approach as with initializing or finalizing the migration.
The user initiates system action (\texttt{Message}, in Elm terminology) which is followed by creating an \gls{api} request to the server.

The action s initiated from migration detail.
Once the server responds, the application redirects user to the list of questionnaires.
After the redirect, application will fetch the latest version of the questionnaires and user will the newer version display alongside the original one.
