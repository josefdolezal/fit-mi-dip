\section{Server-side application}

As mentioned earlier, the server part of the application is written using Haskell programming language using Scotty web framework.
The application exposes public \gls{rest} \glsentryshort{api} over \glsentryshort{http} protocol.

To achieve modularity and lose coupling, the server application is split into multiple modules.
Those modules are grouped into logical partitions based on their purpose.
The most significant parts are:

\begin{itemize}
    \item Handler layer,
    \item Service and \gls{dto} layers,
    \item Model and \gls{dao} layers.
\end{itemize}

\subsection{Handler layer}\label{sec:handler}
A handler is responsible for processing \gls{api} requests.
It directly interacts with Scotty framework, transforms incoming data into strongly typed objects and orchestrates other layers to evaluate response.

Handler groups together multiple endpoints.
Those endpoints are uniquely registered to a specific \gls{url} and \gls{http} method.
In terms of the server application, we refer to such pair as a \textit{route}.

\hscode{code:router-handler}{Routes definition and route endpoint (simplified)}{router-handler.hs}

Once the correct handler for the requested route is selected, it starts processing data from the request.
Such data might be \gls{url} (or query) parameters and/or request body.
Those data are transformed into language primitives (such as \texttt{Int} or \texttt{String}) or complex objects called \gls{dto} (explained later).
In case of invalid or malformed data, the request is aborted immediately with appropriate error information in response.

In addition to that, some routes may also require several permissions in order to be executed.
Those routes need requests to be authenticated using \gls{jwt} technology and user tied with given token must be granted such permissions.

If the request is validated successfully, required data are passed into a service layer where the actual business logic happens.

\subsection{Service layer}
Services are responsible for application logic.
This means that its public interface is exposed to other layers using \gls{dto}.
Private functions and internal dependencies (such as data persistence) are thus \textit{implementation detail} of the layer itself and do not affect its interface.

In \gls{dsw}, service layer takes care of basic \gls{crud} operations over persisted entities.
This includes application objects, for example, \texttt{User}, \texttt{KnowledgeModel} or \texttt{Questionnaire}.

In addition to data persistence, this layer also manages application configuration, knowledge model migrations, \gls{dto} mapping and maintains data consistency.
To keep the public interface simple, services are usually composed using other services or \gls{dao}.

Services make operations based on given input data (objects identifiers, \gls{dto}, \dots).
Operations itself then converts \gls{dto} objects into internal representation (persisted object) and do computations on it.
As a result of all computations, the \gls{dto} object is returned from service.

\subsection{\gls{dto}}
While describing the Service layer earlier in this chapter, there were multiple references to objects called \gls{dto}.
These objects' goal is to create a framework-independent interface between layers.
\gls{dto}s are plain objects\footnote{By plain objects, we usually mean structures, which are not restricted by any framework and are built only constructs available in the language itself or by other plain objects (using composition). The same pattern is available in other languages. In Java programming language, this concept is referred to as \gls{pojo}\cite{wiki-pojo}.} representing standard model objects, usually in many ways.

Model objects may be complex structures containing a significant amount of data.
Such complexity may or may not be appropriate for some service operations.

While presenting a list of data (populated by \gls{dto}) to the user, it is not necessary to fetch all nested objects recursively.
Instead, simplified objects may be used.
Additional data may then be requested on demand.
On the other hand, when the user wants to see the detail of some list item, the \gls{dto} object should contain enough data to fulfill the user's expectation.
For such usecase, there would be two different objects, and the service layer would choose the one which best fits user expectation.

\hscode{code:dto}{DTO definition and transformation from the Model object}{dto.hs}

In \gls{dsw} this approach is used to display either list or detail of questionnaires, knowledge models or knowledge model customizations.
Conversion between model and \gls{dto} objects is done using specific services.

\subsection{Authentication}
In section \ref{sec:handler} about \gls{http} requests handling, I briefly discussed endpoint authentication using \gls{jwt}.
In this section, the reader will be acquainted with how these tokens work.

Even though some endpoints (such as login or registration) do not require the user to be authenticated, the vast majority of the application is based on managing private data.
Therefore, the user needs to be authenticated using username and password before he or she can access these data.
Similarly to data persistence, also authentication may be done in many ways.
There is basic authorization\footnote{Base64 is an HTTP authentication used to encode user’s credentials into request header directly. Such aan pproach is considered as insecure as the credentials are easily captured by an unauthorized person\cite{qnimate-base64}.}, API keys, \gls{jwt} and many more.

The \gls{dsw} application uses \gls{jwt} tokens, which stands out by having the ability to be verified for issuer authenticity.
Thanks to this, the issuer (in this case the server application) can embed custom payload into token and be sure those data will not be modified by an unauthorized person.

To become authorized, the user first has to log in to the application using username and password.
If those credentials are correct, the server will issue an access token and send it back to the user.
For all request requiring authorization, the user has to send issued token in the request header.
The server will read the payload, verify token integrity and check the user's permissions.
If the token is valid, the appropriate handler is called.

All \gls{jwt} tokens have the same structure (pictured in figure \ref{fig:jwt-token}).
The token consists of three parts: \texttt{Header}, \texttt{Payload}, and \texttt{Signature}\cite{jwt-intro}.

\image[0.8\textwidth]{fig:jwt-token}{JWT token structure example}{jwt-example.pdf}

The header contains information about the token type and hashing algorithm used to generate the signature.
The payload is an arbitrary \gls{json} object containing publicly visible data (such as user identifier).
Token signature is created by hashing header and payload (both encoded using the Base64 algorithm) by hashing algorithm specified in the header and also encoded using Base64.

These three parts are combined using period (.) character and returned as a \texttt{String}.

\subsection{Authorization}

In modern application, authenticating users is not enough.
We might want a system to support a wide range of users hierarchy, capabilities and responsibilities.
In \gls{dsw} this is solved using combinations of roles and permissions.

Each user has one of the following roles:

\begin{itemize}
    \item Administrator,
    \item Data Steward,
    \item Researcher.
\end{itemize}

Once the user entity is created, it has assigned a role and a default set of permissions.
Having two levels of authorization is a vital system feature as the user's account might gain or lose privileges over its lifetime.

Permissions are resource specific, and for each resource, we might distinguish between multiple permissions (for example read and write permissions).
Default permissions are system-wide configurable using a configuration file (which is loaded at startup time).

In the default environment, \textit{Administrator} has all possible permissions including management of the users, organizations, and content.
The \textit{Data Steward} -- the role in the hierarchy just under the Administrator -- has similar permissions but is unable to manage organizations and other users.
The \textit{Researcher} can see knowledge models and public questionnaires but is able only to modify the content he created.

As mentioned earlier in section \ref{sec:handler}, permissions are checked in handler before any business logic happens.
Each handler has defined which permission is required (for example \texttt{Read questionnaire}).
Once the user makes a request, handler checks authentication state and search user's permissions (demonstrated in code example \ref{code:router-handler}).
If user the has granted required permissions, request processing continues.
Otherwise, the request is aborted immediately.

\subsection{\gls{dao}}\label{sec:dao}
\gls{dao} is a way to access persisted data using a simplified interface.
In general, application data may be persisted in many ways.
The most common approach to this is using a database\cite{dzone-fs-vs-db}.
However, even database persistence may be implemented in different ways.

On the market, there are several options available.
There are various kinds of databases: Graph databases, Relational databases, Document databases, and many others -- moreover, all of those offered as free to use as well as business solutions.

In \gls{dsw}, the Mongo DB was chosen as a database for persisting data.
Mongo DB is a NoSQL document database (explained later), which allows to storing data using nested structures.

The goal of \gls{dao} is to encapsulate such technical detail from other application layers.
The underlying database may change over time, but as long as the data model stays the same, the only affected layer will be \gls{dao}.

In Haskell, such layer is implemented using independent modules where each module manages one resource (Mongo DB collection).
The module public interface is implemented using free functions.
Those functions offer high-level API such as \texttt{findAll}, \texttt{findById} and similar for all \gls{crud} operations.

\hscode{code:dao}{DAO module example for Questionnaire entity}{dao.hs}

\subsection{Mongo DB}\label{sec:mongo-db}
Mongo DB was chosen as a persistent layer in the early development of \gls{dsw}.
As stated earlier in section \ref{sec:dao}, Mongo DB is a document database.

By document database, we understand system which can store hierarchical tree structures (so-called \textit{documents}) composed using scalars (\texttt{String}, \texttt{Integer}, \dots), hashable maps, arrays or nested documents\cite{aws-document-database}.
These documents are identified using unique internal identifiers and grouped into collections.

On collections, we can run standard \gls{crud} queries to manipulate with stored data.
Mongo DB provides an \gls{api} for querying objects using notation based on \gls{json}.
This notation allows querying on nested objects, aggregations, relations or regular expressions.
Since \gls{json} notation may be unnecessarily verbose, documents are internally stored in \gls{bson} representation\cite{ip-mongo}.

\jscode{code:mongo-find-one}{Mongo DB query for finding questionnaire using its identifier}{mongodb-findOne.js}

In oppose to standard relational databases, collections and documents do not have an explicit schema.
Therefore there might be stored almost any kind of data in collection at one time.

Another notable difference is in data normalization.
Standard databases (based on \gls{sql}) use data decomposition and normalization to achieve better performance and great organization of tables and columns.
In the case of Mongo DB, related documents are usually tightly coupled together (using composition) and duplicated instead of relations using foreign keys.

As mentioned before, the data are internally stored in binary format which is not directly usable in Haskell.
Therefore the database driver exposes two special Typeclasses\footnote{More information about Haskell Typeclasses are available at https://www.haskell.org/tutorial/classes.html} for encoding and decoding binary formatted data.
Such typeclasses must be implemented by all types which will be stored in database.

\hscode{code:bson-mapping}{Example of Typeclasses used to transform Model object into binary representation}{haskell-bson-mapping.hs}

In addition to coding typeclasses, the driver also exposes a type-safe interface to build \gls{crud} queries.
These queries are created using simple \gls{dsl} as shown in example \ref{code:dao}.
