\subsection{DSW-Server}\label{sec:dsw-server}

The backend part of the application is written using the Haskell programming language.
The primary goal of this portal is to help researchers create data management plans for their experiments.
To be able to create a management plan, at least one knowledge model must be created.
At the time of writing, the \gls{dsw} is shipped without any prebuild knowledge model by default.
The maintaing team, however, provides the \textit{core} knowledge model separately to help data stewards setup initial data faster\cite{gh-km-core}.

Public knowledge models (such as \textit{core}) are represented using \gls{json} and stored in a file on a public server.
Data stewards then use these models to create questionaries.
The role of researchers is to fill in the questionaries with meta information about their research.

Filled-in questionaries are used by data stewards to create management plans.
This flow mostly covers system features.

We, however, have silent assumptions in the flow described above.
The portal also has to support the following features in addition to the mentioned business logic:

\begin{itemize}
    \item User management (roles, authorization, and authentication),
    \item Knowledge model (and its customizations) administration including creation, editing, and versioning,
    \item Data persistence (database connection, file system integration),
    \item \glsentryshort{rest} \glsentryshort{api}.
\end{itemize}

All of the mentioned functions are split into independent Haskell modules.
The \glsentryshort{rest} \glsentryshort{api} layer is built using Scotty framework\footnote{Scotty is an open-source web framework written in Haskell, inspired by Ruby’s Sinatra. It is a lightweight alternative to frameworks like Yesod or Spock. The source code and more information is available on the project homepage: https://github.com/scotty-web/scotty.}.
