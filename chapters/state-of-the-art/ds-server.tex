\subsection{DSW-Server}\label{sec:dsw-server}

The backend part of the application is written using the Haskell programming language.
The primary goal of this portal is to create data management plans. \reword{Most of the sentences starts with the word "the", reword it so it does not feel as machine generated.}
At least one knowledge model must be added to the system before data plans can be generated.
At the time of writing, \gls{dsw} is shipped with a prebuild knowledge model called \textit{core}.

Knowledge models are represented using \gls{json} and stored in a file on the server file system.
Data stewards then use these models to create questionaries.
The role of researchers is to fill in the questionaries with meta information about their research.

Filled-in questionaries are used by data stewards to create management plans.
This flow mostly covers system features.

We, however, have silent assumptions in the flow described above.
The portal also has to support the following features in addition to the mentioned business logic:

\begin{itemize}
    \item User management (roles, authorization and authentication),
    \item Package administration (cration, editing and versioning),
    \item Data persistency (database connection, file system integration)
    \item \glsentryshort{rest} \glsentryshort{api}.
\end{itemize}

All of the mentioned functions are split into independent Haskell modules.
The \glsentryshort{rest} \glsentryshort{api} layer is built using Scotty framework\footnote{Scotty is an open-source web framework written in Haskell, inspired by Ruby's Sinatra. It is a lightweight alternative to frameworks like Yesod or Spock. The source code is available on GitHub: https://github.com/scotty-web/scotty.}.
