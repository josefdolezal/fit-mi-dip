\section{Knowledge model migrations}\label{sec:km-migrations}

Initially, the \gls{dsw} portal was created as an user-friendly alternative to knowledge model management which was previously built using \gls{json} notation.
This notation was originally created and maintained by Rob Hooft\footnote{Rob Hooft is an manager of Netherland node of ELIXIR group (ELIXIR NL). Mr. Hooft created the initial knowledge model data source which is currently used in \gls{dsw} as an optional component.} and is still used as core knowledge model in \gls{dsw}.

As knowledge models may change over time, it is required to keep track of all possible versions created in past.
In terms of \gls{dsw}, such process of modifying knowledge model is called migration.
The migration process was designed and implemented by Ing. Vojtěch Knaisl as a part of his master thesis\cite{mt-knaisl}.

Migration consists of two parts: modification and upgrade.
In the modification part, data steward may do \gls{crud} operations over all knowledge model nodes (this includes chapters, questions, answers, references and experts).
Once the modifications are done, data steward publish new version of knowledge model so the modifications are available to other stewards.

The second part, upgrade, is done on knowledge model customizations.
These customizations may be for example localizations, which needs to reflect changes in its parent knowledge model.
During upgrade, system asks data steward step by step for all knowledge model chagnes comming from parent knowledge model.
Since modifications on a single node may be done on both customizations and parent knowledge model, conflicts may occure.

In case of conflict, user is asked to solve it manually.
There are two options for user to solve conflict.
User may either accept incomming change (which will discard customization changes) or reject incomming change (which will in oppose prefer customization changes).
Changes which do not cause conflict are merged automatically without user interaction.

Once all conflicts are resolve, data steward is asked to complete migration by publishing new version of customization.

On top of knowledge models, reasearchers creates and fills questionnaires which are later used to create data management plans.
Questionnaires are currently tighly coupled to knowledge model they were initially created on.
This however means, that once new version of knowledge model is released, new questionnaire has to be created and filled from scratch.

As a result of this master's thesis, researchers will be able to migrate their questonnaires to newer version similary as data stewards are able to migrate knowledge models.

\subsection{Event-driven architecture}\label{sec:event-driven-architecture}

In previous section \ref{sec:km-migrations}, I discussed the idea behind having multiple versions of the same data source at once.
In this section, I would like to briefly describe the technical details of that idea.

In system representation, knowledge models are nothing more than sequence of modifiation events.
Those events are strictly defined and must be always performed in order they were created.
Single event represents one specific modification.
This modification may be "Add chapter", "Add question" or "Remove answer".
Together, system supports exactly twenty events where each of these events contains additional meta data.
In Haskell, such structure is represented using multiple constructors of the same data type (example \ref{code:events}).

\hscode{code:events}{Event representation in \gls{dsw}}{events.hs}

This approach is generally known as \textit{event-driven architecture}\cite{mdm-event-architecture}.
One of the main benefits for \gls{dsw} is that by repeatedly applying same events again, we always get same result.

This works well with Mongo DB as persistence layer.
Instead of compiling (by applying events) knowledge models and storing result in databse multiple times (for each custoization, questionnaire, \dots), only events are stored.
When user requests compiled knowledge model, it is always compiled on demand presented using dto.
Thanks to that, user always has the latest possible version without problems with inconsistency which would be possible over time.

Another great feature of chain of events is that it can be easily manipulated.
For example, merging two customizations of knowledge model may be easily done by applying one set of events on the end of the other chain.
It also allows user to "cherry pick" events and apply only subset of incomming changes.
Those properties are greatly utilized in knowledge model migrations mentioned earlier.