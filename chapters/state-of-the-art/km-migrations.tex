\section{Knowledge model migrations}\label{sec:km-migrations}

Initially, the \gls{dsw} portal was created as a user-friendly alternative to knowledge model management which was previously built using \gls{json} notation.
This notation was initially created and maintained by Rob Hooft\footnote{Rob Hooft is a manager of Netherland node of ELIXIR group (ELIXIR NL). Mr. Hooft created the original knowledge model data source which is currently used in \gls{dsw} as an optional component.} and is still used as core knowledge model in \gls{dsw}.

As knowledge models may change over time, it is required to keep track of all possible versions created in the past.
In terms of \gls{dsw}, such process of modifying knowledge model is called migration.
The migration process was designed and implemented by Ing. Vojtěch Knaisl as a part of his master thesis\cite{mt-knaisl}.

Migration consists of two parts: modification and upgrade.
In the modification part, data steward may do \gls{crud} operations on all knowledge model nodes (this includes chapters, questions, answers, references, and experts).
Once the modifications are done, data steward publish a new version of the knowledge model, so the modifications are available to other stewards.

The second part, upgrade, is done on knowledge model customizations.
These customizations may be for example localizations, which needs to reflect changes in its parent knowledge model.
During the upgrade, the system asks data steward step by step for all knowledge model changes coming from parent knowledge model.
Since modifications on a single node may be done on both customizations and parent knowledge model, conflicts may occur.

In case of conflict, the user is asked to solve it manually.
There are two options for the user to solve a conflict.
User may either accept incoming change (which will discard customization changes) or reject the incoming change (which will in oppose prefer customization changes).
Changes which do not cause conflict are merged automatically without user interaction.

Once all conflicts are resolved, data steward is asked to finalize the migration by publishing new a version of customization.

On top of knowledge models, researchers create and fill questionnaires which are later used to create data management plans.
Questionnaires are currently tightly coupled to knowledge model they were initially created on.
This, however, means, that once a new version of the knowledge model is released, a new questionnaire has to be created and filled from scratch.

As a result of this master's thesis, researchers will be able to migrate their questonnaires to newer version similarly as data stewards can migrate knowledge models.

\subsection{Event-driven architecture}\label{sec:event-driven-architecture}

In previous section \ref{sec:km-migrations}, I discussed the idea behind having multiple versions of the same data source at once.
In this section,  I would like to describe the technical details of that idea briefly.

In system representation, knowledge models are nothing more than a sequence of modification events.
Those events are strictly defined and must always be performed in the order they initially were created.
The single event represents one specific modification.
This modification may be "Add chapter", "Add question" or "Remove answer".
Together, the system supports exactly twenty events where each of these events contains additional metadata.
In Haskell, such structure is represented using multiple constructors of the same data type (example \ref{code:events}).

\hscode{code:events}{Event representation in \gls{dsw}}{events.hs}

Such an approach is generally known as \textit{event-driven architecture}\cite{mdm-event-architecture}.
One of the main benefits for \gls{dsw} is that by repeatedly applying the same events again, we always get the same result.

This works well with Mongo DB as a persistence layer.
Instead of compiling (by applying events) knowledge models and storing the result in the database multiple times (for each customization, questionnaire, \dots), only events are stored.
When user requests compiled knowledge model, it is always compiled on demand and presented using \gls{dto}.
Thanks to that, the user always has the latest possible version without problems with the inconsistency which would not be possible over time otherwise.

Another great feature of a chain of events is that it can be easily manipulated.
For example, merging two customizations of knowledge model may be quickly done by applying one set of events at the end of the other chain.
It also allows the user to \textit{cherry pick} events and apply only a subset of incoming changes.
Those properties are greatly utilized in the knowledge model migrations mentioned earlier.
