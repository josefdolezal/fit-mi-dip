\section{Frontend application}\label{sec:frontend-application}

In this section, I will in detail discuss the approach of development of the frontend (client) application.
As stated earlier in section \ref{sec:dsw-client}, this part of the application is written using functional language Elm.

Even though Haskell and Elm have similar syntax, the application structure is entirely different.
Elm application entry point is a module called \texttt{Main}.
This module's responsibility is to initialize the application in the browser window and set its state based on the current browser \gls{url}.

The application view is a function of the state.
Every time the state changes, the view function is called with the latest state value and returns corresponding \gls{ui} elements.
The resulting view is then passed to Elm runtime.

The runtime will compare the given view with current \gls{dom}\footnote{HTML DOM is a tree of web document (page) objects. It is used to create dynamic HTML modifiable by JavaScript. More information about DOM is available at W3C documentation: https://www.w3schools.com/js/js\_htmldom.asp.} and applies only appropriate modifications.
Since rendering the whole \gls{dom} may be slow\cite{accelebrate-slow-dom}, Elm may group multiple changes and render them at once.
This means the browser does not have to render the whole tree so often and the user experience may increase.

For an extensive application as \gls{dsw} is, it is not possible to keep the architecture that simple.
Instead, for each screen new module with the same architecture is created.
In the rest of this thesis, I will refer to such modules as "subapplicatons" for clarity.
This means that state update, view functions, and messages are created from scratch for each screen.
To accomplish interoperation between subapplications and standard Elm architecture, the composition is used.

In this case, the composition means that every nested application screen state is managed by its superior screen.
A similar idea is used for \texttt{update} function where superior screen calls nested's screen update function.

In the time of writing, the client is composed of six subapplications, namely: \texttt{Users}, \texttt{Questionnaires}, \texttt{KMEditor}, \texttt{KnowledgeModels}, \texttt{Organization}, and \texttt{Public}.
In the following sections, I will shortly introduce the reader to each of these.

\subsection{Users module}\label{sec:users-module}

This module is primarily used by the system administrator.
The administrator can list registered users, manage their profiles and roles.

For each user, the administrator may update the user's profile information (including email) and change password.
The user also may be deleted or deactivated in order not to be able to use application furthermore.

The rest of the users are only allowed to update their profiles without being able to deactivate or delete it.

\subsection{Questionnaires module}\label{sec:questionnaires-module}

Questionnaires module has two parts available to all user roles.
As each of those parts contains complex logic (such as network calls and state management), it is also separated into individual subapplications.

The first part is a list of existing questionnaires with the ability to create a new one.
Each user can see either public questionnaires or private questionnaires he or she created.
By selecting a questionnaire, user can see questionnaire detail.

The second part is a possibility to fill a questionnaire and generate the data management plan.
User fills the questionnaire by answering a set of questions from the knowledge model which was selected during the process of creating the questionnaire.
Some of the questions have assigned \gls{fair} metrics.
By selecting an answer to a question, its metrics may positively or negatively affect the overall questionnaire.

Since researched project state may change over time, the user can change its phase on the questionnaire detail page.
As some questions are only desired in a specific project phase, this action will also probably affect set questions which need to be answered.

Questions are designed to be infinitely nested and composed using other questions.
Nested questions are in terms of \gls{dsw} called \texttt{Item Templates}.
In order to answer such question, the user is required to create item answer and reply to all nested questions in it.

Another type of nested question is follow-up questions for answer item (from the single choice list).
The application is designed to make follow-up questions optional for items which do not require it.
Since questionnaires may quickly become complicated and hard to orientate in, follow-up questions are not visible until the appropriate answer is selected.

\subsection{KMEditor module}\label{sec:kmeditor-module}

In section \ref{sec:km-migrations} about knowledge models migration, I will describe how editing knowledge models is designed and implemented.
In this section, I will describe how the \gls{ui} of customizing knowledge models work.

\texttt{KMEditor module} is mainly designed for data stewards (even though it is available to other roles too).
Stewards are allowed to list existing knowledge models customizations or create a new one.
We can understand customization as a new branch of existing (or even new) knowledge model which contains changes from the superior knowledge model.

Together with listing existing customizations, the user can also create a new one.
When creating a new one, the user is asked to either select superior knowledge model or start from scratch.
In both options, a new screen is displayed with details of customization.

The editor allows managing the whole knowledge model structure mentioned earlier: Chapters, Questions, Answers, References, and Experts.
All of these may be added, modified or deleted.
Changes are synchronized to the server in a batch once the user decides to save the current version of the knowledge model explicitly.

Nodes overview is displayed in a hierarchical tree view where modified nodes are highlighted using appropriate color.
Navigation in the knowledge model is done using either tree view on the left side of the screen or from the selected node overview on the right side.

Node overview displays configuration specific to each node type together with its nested nodes.

During the time of writing this thesis, new capabilities for the editor are currently in development by its maintainers.
The first one allows to tag questions.
It helps to categorize questions into logical groups by them.
Later, while creating a new questionnaire, data steward can pick only questions having specified tags, which are  relevant for the questionnaire.

The second one is knowledge model preview.
Currently, there is no way to quickly preview how questionnaire built on edited knowledge model will look like.
To do that, the user is required to save and publish the knowledge model and then create a questionnaire to see the \gls{ui}.
In the new version, users will be able to see an overview of final then questionnaire right in the editor without a need to publish unfinished customization.

The editor \gls{ui} is shown in figure \ref{fig:kme-screen}.

\image[0.9\textwidth]{fig:kme-screen}{The knowledge model editor}{kme-screen.png}

\subsection{Knowledge models module}

This module partially groups features from both \texttt{Questionnaires} and \texttt{KMEditor} modules.
The main difference is, that this screen directly operates with knowledge models, instead of customizations.
This however is not a significant difference because all published customizations are listed in knowledge models too.
For the rest of the features, all users may see knowledge models with its detail, create a questionnaire or new customization.

The last feature is related to system interoperability.
Users are allowed to both import new knowledge models or export an existing one in \gls{json} format.

This is an essential feature as base knowledge may, in theory, be created by anyone -- either by individuals or enterprises.
Such models then could be shared publicly and used in arbitrary instance of \gls{dsw}.

Even today, importing models is very useful feature.
The default installation package of \gls{dsw} is currently shipped without any content.
However, the initial base model (described later in section \ref{sec:km-migrations}) maintained by Mr. Rob Hooft is shared publicly and can be easily imported.

\subsection{Organization module}

Organization management is a simple module which allows the administrator to manage necessary information (currently only name and identifier) of an organization operating the given instance of the application.
This information is mostly used to identify knowledge models created by the organization and to add metadata to exported data management plans.

\subsection{Public module}

Public module groups all subapplications which are accessible without being logged in to the system.
Those modules are:

\begin{itemize}
    \item Login,
    \item Registration,
    \item Public questionnaire.
\end{itemize}

The login module is predominantly self-explanatory.
It is and landing page for not-logged-in users who are prompted to enter credentials.
After a successful login, the server will issue a \gls{jwt} access token which will authorize the user to see private pages.

Registration module is similarly straightforward.
It is designed for users who are not signed up to the system yet and wants to use it.
After successful registration, the user is required to confirm his email address and activate the account.
Once the profile is activated, the user able to log in to the system.

Last part is a public questionnaire.
This serves as an application demo which allows unauthenticated researchers to fill a prepared questionnaire.
Since the public questionnaire is for demonstration purpose only, its data answers are not stored anywhere and will be forgotten once the user leaves the screen.

\image[0.7\textwidth]{fig:system-architecture}{The system architecture}{analysis-system-architecture.pdf}
