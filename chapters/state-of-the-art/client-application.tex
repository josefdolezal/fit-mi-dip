\section{Frontend application}\label{sec:frontend-application}

In this section, I will in detail discuss approach of development of the frontend (client) application.
As stated earlier in section \ref{sec:dsw-client}, this part of the application is written using functional language Elm.

Even though Haskell and Elm have similar syntax, the application structure is completely different.
Elm application entrypoint is an module called \texttt{Main}.
This module's responsibility is to initialize the application in browser window and set its state based on current browser \gls{url}.

The application view is basically function of state.
Every time the state changes, the view function is called with latest state value and returns corresponding \gls{ui} elements.
The resulting view is then passed to Elm runtime.

Runtime will compare the given view with current \gls{dom}\footnote{HTML DOM is a tree of web document (page) objects. It is used to create dynamic HTML modifieable by JavaScript. More information about DOM are avaialbe at W3C documentation: https://www.w3schools.com/js/js\_htmldom.asp.} and applies only appropriate modifications.
Since rendering the whole \gls{dom} may be slow\cite{accelebrate-slow-dom}, Elm may group multiple changes and render them at once.
This means browser does not have to render the whole tree so often and the user experience may increase.

For as extensive application as \gls{dsw} is, it is not possible to keep the architecture that simple.
Instead, for each screen new module with same architecture is created.
In the rest of this thesis, I will refer to such modules as "subapplicatons" for clarity.
This means that state update, view functions and messages are created from scratch for each screen.
To accomplish intercoporation between subapplications and standard Elm architecture, composition is used.

What is meant by composition is that every nested application screen state is managed by superior screen.
Similar idea is used for update function where superior screen calls nested's screen update function.

In the time of writing, client is composed from six subapplications, namely: \texttt{Users}, \texttt{Questionnaires}, \texttt{KMEditor}, \texttt{KnowledgeModels}, \texttt{Organization} and \texttt{Public}.
In following sections, I will shortly introduce reader to each of these.

\subsection{Users module}\label{sec:users-module}

This module is primarily used by system administrator.
Administrator is able to list registred users, manage their profiles and roles.

For each user, administrator may update user's profile information (including email) and change password.
User also may be deleted or deactivated in order to not be able to use application further more.

The rest of users are only allowed to update own profiles without being able to deactivate or delete it.

\subsection{Questionnaires module}\label{sec:questionnaires-module}

Questionnaires module has two parts available to all user roles.
As each of those parts contains complex logic (such network calls and state management), it's also separated into individual subapplications.

The first part is a list of existing questionnaires with ability to create new one.
Each user is able to see either public questionnaires or private questionnaires he or she created.
By selecting questionnaire, user can see questionnaire detail.

The second part is possibility to fill questionnaire and generate data management plan.
User fills questionnaire by answering set of questions from knowledge model which was selected during process of creating the questionnaire.
Some of the questions have assigned \gls{fair} metrics.
By selecting an answer for question, its metrics may positively or negatively affect the overall questionnaire.

Since researched project state may change over time, user is able to change its phase in questionnaire.
As some questions are only desired in a specific project phase, this action will also probably affect set questions which needs to be answered.

Questions are designed to be infinitely nested and composed using other questions.
Nested questions are in terms of \gls{dsw} called \texttt{Item Templates}.
In order to answer such question, user is required to create item answer and reply to all nested questions in it.

Another type of nested question are follow-up questions for answer item (from single choice list).
The application is designed to make follow-up questions optional for items which does not require it.
Since questionnaires may easily become complex and hard to orientate, follow-up questions are not visible until appropriate answer is selected.

\subsection{KMEditor module}\label{sec:kmeditor-module}

In section \ref{sec:km-migrations} about knowledge models migration, I will describe how editing knowledge models is designed and implemented.
In this section, I will describe how the \gls{ui} of customizing knowledge models work.

\texttt{KMEditor module} is mainly designed for data stewards (even though it is available to other roles too) .
Stewards are allowed to list existing knowledge models customizations or create a new one.
We can understand customization as a new branch of existing (or even new) knowledge model which contains changes from the superior knowledge model.

Together with listing existing customizations, user is able to create a new one.
When creating a new one, user is asked to either select superior knowledge model or start from scratch.
In both options, new screen is displayed with details of customization.

Editor allows to manage the whole knowledge model structure mentioned earlier: Chapters, Questions, Answers, References and Experts.
All of these may be added, modified or deleted.
Changes are synchronized to the server in a batch once the user decides to explicitly save current version of knowledge model.

Nodes overview is displayed in a hierarchical tree view where modified nodes are highlighted using appropriate color.
Navigation in knowledge model is done using either tree view on the left side of the screen or from selected node overview on the right side.

Node overview displays configuration specific to each node type together with its nested nodes.

During time of writing this thesis, new capabilities for editor are currently in development its maintainers.
The first one allows to tag questions.
This helps categorize questions into logical groups by its.
Later, while creating new questionnaire, data steward is able to pick only questions having specified tags, which are for the questionnaire valid.

The second one is knowledge model preview.
Currently, there is no way to quickly preview how questionnaire built on edited knowledge model will looklike.
To do that, user is required to save and publish the knowledge model and thne create questionnaire to see the \gls{ui}.
In new version, users will be able to see overview of final questionnaire right in the editor without a need to publish unfinished customization.

\todo{Add KMEditor screenshot}

\subsection{Knowledge models module}

This module partialy groups features from both \texttt{Questionnaires} and \texttt{KMEditor} modules.
The main difference is, that this screen directly operate with knowledge models, instead of customizations.
This however is not a significant difference because all published customizations are listed in knowledge models too.
For the rest of the features, all users may see knowledge models with its detail, create questionnaire or new customization.

The last feature is related to system interoperability.
Users are allowed to both import new knowledge models or export existing in \gls{json} format.

This is important feature as base knowledge may in theory be created anyone -- either by individuals or by enterprises.
Such models then could be shared publicly and used arbitrary instance of \gls{dsw}.

Even today, importing models is very useful feature.
The default installation package of \gls{dsw} is currently shiped without any content.
However, the intial base model (described later in section \ref{sec:km-migrations}) maintained by Mr. Rob Hooft is shared publicly and can be easily imported.

\subsection{Organization module}

Organization management is a simple module which allows administrator to manage basic information (currently only name and identifier) of organization operating given instance of the application.
These informations are mostly used to identify knowledge models created by the organization and to add metadata to exported data management plans.

\subsection{Public module}

Public module groups all subapplications which are accessible without being logged in to the system.
Those modules are:

\begin{itemize}
    \item Login,
    \item Registration,
    \item Public questionnaire.
\end{itemize}

Login module is predominantly self-explanatory.
It is and landing page for not logged-in users who are prompted to enter credentials.
After successfull login, server will issue \gls{jwt} access token which will authorize user to see private pages.

Registration module is similary simple.
It is designed for users who are not signed up to the system yet and wants to use it.
After successfull registration, user is required to confirm his email address and activate account.
Once the profile is activated, user able to to log in to system.

Last part is a public questionnaire.
This servers as an application demo which allows unauthenticated researchers to fill prepared questionnaire.
Since the public questionnaire is for demonstration purpose only, its data answers are not stored anywhere and will be forgotten once user leaves the screen.

\image[0.5\textwidth]{fig:system-architecture}{System architecture}{analysis-system-architecture.pdf}
