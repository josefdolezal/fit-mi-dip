\section{Deployment}

In previous sections, I described the architecture of the \gls{dsw} distributed system.
This section is dedicated to the production environment of this application.

Since both parts of the system are independent, deployment of the whole application is done in two steps.
In one step, the server side application is deployed.
As discussed earlier, Haskell application does not need application server (the requests are handled directly by the application) but depends on the persistence layer.
To make the deployed application lose coupled, two Docker containers\footnote{Docker is a development and deployment tool using containers to package up an application (together with its dependencies) and deploy it as a single package. Containers utilize hosts OS kernel instead of creating virtual a machine and virtual OS \cite{ms-docker}.} are used to decouple application itself from the database.
To interconnect the containers, Docker compose is used to make a private network between them.
From the user perspective, the application behaves as a monolith which internally encapsulates its complexity.

In the second step, the client application is deployed similarly.
Since Elm application runs in the user's internet browser, there is an additional container which handles \gls{http} requests and servers the application to the user.
This container may be an arbitrary \gls{http} server.
For the purpose of \gls{dsw}, the Nginx HTTP server was chosen.
Nginx is configured to return an empty \gls{http} document which links compiled Elm application.
The actual logic (including routing) is handled locally in the browser.

The deployment schema is shown in figure \ref{fig:deployment-schema}.

\image[0.8\textwidth]{fig:deployment-schema}{The application deployment schema}{deployment-archi.pdf}
