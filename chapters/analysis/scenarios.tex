\section{Scenarios}\label{sec:scenarios}

In previous sections, I described which requirements are relevant to questionnaire migrations.
Requirements were validated against use cases.
In this section, the reader will be further acquainted with use case execution demonstrated on scenarios.

Scenarios are presented in a step-by-step form where each step represents interaction with system.

\subsection{Scenarios actors}\label{sec:scenarios-actors}

Before describing individual scenarios, I will briefly introduce actors participating in these scenarios.
Scenarios has following actors:

\begin{itemize}
    \item \textbf{User}

    An actor interacting with \textit{system} in a role of an arbitrary authenticated user.

    \item \textbf{Application}
    
    An actor representing both client and server side of the system.
    Acts as a blackbox from client perspective.
    Its tasks may internally consist of communication between these two parts.

\end{itemize}

\subsection{Scenarios description}\label{sec:scenarios-description}

This sections contains descriptions for individual scenarios.
Roles described in previsous section will be highlighted using \textit{italic} font.

All use cases starts in implicit state where user is authenticated and located on a screen with list of questionnaires.

\requirement{Scenario for UC1 -- Create questionnaire migration}{%
    \begin{enumerate}
        \item The \textit{user} is notified about available migration.
        \item The \textit{user} selects and upgrade option offered by \textit{application}.
        \item The \textit{application} prompts user to select which of available versions of the knowlegdge model wants to migrate to.
        \item The \textit{user} selects desired version.
        \item The \textit{user} confirms selection.
        \item The \textit{application} creates questionnaire migration and shows it to the \textit{user}
    \end{enumerate}

    \begin{addmargin}[2em]{0em}
        \begin{description}
            \item[Output state:] The migration process is started and the \textit{user} sees preview of a first change.
        \end{description}
    \end{addmargin}
}

\requirement{Scenario for UC2 -- Cancel questionnaire migration}{%
    \begin{addmargin}[2em]{0em}
        \begin{description}
            \item[Input state:] The \textit{user} created an migration as described in UC1.
        \end{description}
    \end{addmargin}

    \begin{enumerate}
        \item The \textit{application} notifies user that there is a migration in progress on one of the \textit{user's} questionaire.
        \item The \textit{user} selectes option "Cancel migration" displayed along with other questionnaire actions.
        \item The \textit{application} will discard all data related to the questionnaire.
        \item The \textit{user} is notified that there is an upgrade available for the questionnaire.
    \end{enumerate}

    \begin{addmargin}[2em]{0em}
        \begin{description}
            \item[Output state:] The \textit{user} reverted questionnaire to the intial state.
            The \textit{user} is at same state as at the beginning of the scenario for UC1.
        \end{description}
    \end{addmargin}
}

\requirement{Scenario for UC3 -- Display migration change context}{%
    \begin{addmargin}[2em]{0em}
        \begin{description}
            \item[Input state:] The \textit{user} created questionnaire migration according to scenario for UC1.
        \end{description}
    \end{addmargin}

    \begin{enumerate}
        \item The \textit{application} displays detail of a first change in the migration.
        \item The \textit{user} expands hierarchical structure overview.
        \item The \textit{application} displays detailed hierarchy structure and highlights currently displayed change.
        \item The \textit{user} navigates through to the hierarchy to see the context of the change.
    \end{enumerate}
}

\requirement{Scenario for UC4 -- Change migrated question state}{%
    \begin{addmargin}[2em]{0em}
        \begin{description}
            \item[Input state:] The \textit{user} created questionnaire migration according to scenario for UC1.
        \end{description}
    \end{addmargin}

    \begin{enumerate}
        \item The \textit{user} selects option "Continue" migration next to other questionnaire actions.
        \item The \textit{applicaton} displays questionnaire migration process focused on the first change.
        \item The \textit{user} navigates through list of changes until he sees first change of either question or its answer.
        \item The \textit{application} offers to either resolve change or mark it for later review.
            \begin{enumerate}[(a)]
                \item The \textit{user} selects resolve change.
                \item The \textit{user} selects review change later.
            \end{enumerate}
        \item The \textit{application} saves the state and notifies user about new state.
    \end{enumerate}

    \begin{addmargin}[2em]{0em}
        \begin{description}
            \item[Output state:] The \textit{application} stored the updated state so the \textit{user} will see the new state next time he use the application.
            Such state may be undone.
        \end{description}
    \end{addmargin}
}

\requirement{Scenario for UC5 -- Migrate questionnaire}{%
    \begin{addmargin}[2em]{0em}
        \begin{description}
            \item[Input state:] The \textit{user} updated the questionnaire migration state according to UC4 several times.
        \end{description}
    \end{addmargin}

    \begin{enumerate}
        \item The \textit{user} selects option "Continue" migration next to other questionnaire actions.
        \item The \textit{applicaton} displays questionnaire migration process focused on the first change.
        \item The \textit{user} selects "Apply migration" action.
        \item The \textit{application} creates s new copy of the questionnaire on the new version of knowledge model.
        \item The \textit{application} displays list of questionnaires with the original questionnaire on the old version of knowledge model and a new questionnaire on the version.
    \end{enumerate}

    \begin{addmargin}[2em]{0em}
        \begin{description}
            \item[Input state:] The \textit{user} has one more questionnaire in the list of questionnaires.
        \end{description}
    \end{addmargin}
}
