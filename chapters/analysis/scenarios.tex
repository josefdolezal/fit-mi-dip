\section{Scenarios}\label{sec:scenarios}

In previous sections, I described which requirements are relevant to questionnaire migrations.
Requirements were validated against use cases.
In this section, the reader will be further acquainted with use case execution demonstrated on scenarios.

Scenarios are presented in a step-by-step form where each step represents interaction with the system.

\subsection{Scenarios actors}\label{sec:scenarios-actors}

Before describing particular scenarios, I will briefly introduce actors participating in these scenarios.
Scenarios have the following actors:

\begin{itemize}
    \item \textbf{User}

    An actor interacting with them \textit{system} in the role of an arbitrary authenticated user.

    \item \textbf{Application}
    
    An actor who represents both client and server side of the system.
    It acts as a black box from the client perspective.
    Its tasks may internally consist of communication between these two parts.

\end{itemize}

\subsection{Scenarios description}\label{sec:scenarios-description}

This section contains descriptions for individual scenarios.
Roles described in the previous section will be highlighted using the \textit{italic} text style.

All use cases start in an implicit state where the user is authenticated and located on a screen with a list of questionnaires.

\requirement{The scenario for UC1 -- Initiate questionnaire migration}{%
    \begin{enumerate}
        \item The \textit{user} is notified about available migration.
        \item The \textit{user} selects an upgrade option offered by the \textit{application}.
        \item The \textit{application} prompts the \texttt{user} to select which of the available versions of the knowledge model wants to migrate to.
        \item The \textit{user} selects the desired version.
        \item The \textit{user} confirms the selection.
        \item The \textit{application} creates questionnaire migration and shows it to the \textit{user}
    \end{enumerate}

    \begin{addmargin}[2em]{0em}
        \begin{description}
            \item[Output state:] The migration process is started and the \textit{user} sees a preview of a first change.
        \end{description}
    \end{addmargin}
}

\requirement{The scenario for UC2 -- Cancel questionnaire migration}{%
    \begin{addmargin}[2em]{0em}
        \begin{description}
            \item[Input state:] The \textit{user} created a migration as described in the UC1.
        \end{description}
    \end{addmargin}

    \begin{enumerate}
        \item The \textit{application} notifies the user that there is a migration in progress on one of the \textit{user's} questionnaire.
        \item The \textit{user} selects the option "Cancel migration" displayed along with other questionnaire actions.
        \item The \textit{application} will discard all data related to the questionnaire.
        \item The \textit{user} is notified that there is an upgrade available for the questionnaire.
    \end{enumerate}

    \begin{addmargin}[2em]{0em}
        \begin{description}
            \item[Output state:] The \textit{user} reverted questionnaire to the initial state.
            The \textit{user} is at the same state as at the beginning of the scenario for UC1.
        \end{description}
    \end{addmargin}
}

\requirement{The scenario for UC3 -- Display migration change context}{%
    \begin{addmargin}[2em]{0em}
        \begin{description}
            \item[Input state:] The \textit{user} created questionnaire migration according to the scenario for UC1.
        \end{description}
    \end{addmargin}

    \begin{enumerate}
        \item The \textit{application} displays detail of a first change in the migration.
        \item The \textit{user} expands the hierarchical structure overview.
        \item The \textit{application} displays a detailed hierarchy structure and highlights currently displayed change.
        \item The \textit{user} navigates through to the hierarchy to see the context of the change.
    \end{enumerate}
}

\requirement{The scenario for UC4 -- Change migrated question state}{%
    \begin{addmargin}[2em]{0em}
        \begin{description}
            \item[Input state:] The \textit{user} created questionnaire migration according to the scenario for UC1.
        \end{description}
    \end{addmargin}

    \begin{enumerate}
        \item The \textit{user} selects the option "Continue" migration next to other questionnaire actions.
        \item The \textit{application} displays the questionnaire migration process focused on the first change.
        \item The \textit{user} navigates through a list of changes until he sees the first change of either question or its answer.
        \item The \textit{application} offers to either resolve change or mark it for later review.
            \begin{enumerate}[(a)]
                \item The \textit{user} selects the \textit{resolve change option}.
                \item The \textit{user} selects the \textit{review change later} option.
            \end{enumerate}
        \item The \textit{application} saves the state and notifies the user about the new state.
    \end{enumerate}

    \begin{addmargin}[2em]{0em}
        \begin{description}
            \item[Output state:] The \textit{application} stored the updated state so the \textit{user} will see the new state next time he uses the application.
            Such state may be undone.
        \end{description}
    \end{addmargin}
}

\requirement{The scenario for UC5 -- Finalize questionnaire migration}{%
    \begin{addmargin}[2em]{0em}
        \begin{description}
            \item[Input state:] The \textit{user} updated the questionnaire migration state according to UC4 several times.
        \end{description}
    \end{addmargin}

    \begin{enumerate}
        \item The \textit{user} selects the option "Continue migration" next to other questionnaire actions.
        \item The \textit{application} displays the questionnaire migration process focused on the first change.
        \item The \textit{user} selects "Apply migration" action.
        \item The \textit{application} creates a new copy of the questionnaire on the new version of knowledge model.
        \item The \textit{application} displays a list of questionnaires with the original questionnaire (build on the old version of the knowledge model) and a new questionnaire (build on the version).
    \end{enumerate}

    \begin{addmargin}[2em]{0em}
        \begin{description}
            \item[Input state:] The \textit{user} has one more questionnaire in the list of questionnaires.
        \end{description}
    \end{addmargin}
}
