\section{Use cases}\label{sec:use-cases}

Based on identified requirements from the previous section, I would like to further discuss application use cases covering these requirements.
The use cases are initiated by authenticated users without the need for increased permissions.
All identified use cases together with short description are listed below.


\requirement{UC1 -- Initiate questionnaire migration}{%
Enables user to start the questionnaire migration process in case he is notified about new knowledge model version availability.
}

\requirement{UC2 -- Cancel questionnaire migration}{%
Enables user to discard process of the questionnaire migration without affecting the state of the initial questionnaire or its replies.
}

\requirement{UC3 -- Display migration change context}{%
Enables user to navigate through the new knowledge model version in the context of filled in questionnaire replies.
}

\requirement{UC4 -- Change migrated question state}{%
Enables user to modify changed question state to one of the following states:

\begin{itemize}
    \item Initial state -- changed question without state modification,
    \item Needs review -- a change needs to be reviewed after the migration is finalized,
    \item Resolved -- a change does not affect the user's previous answer.
\end{itemize}
}

\requirement{UC5 -- Migrate questionnaire}{%
Enables user to migrate questionnaire to the new version of the knowledge model.
}

\subsection{Requirements coverage}\label{sec:requirements-coverage}

Requirements coverage ensures that all functional requirements identified in section \ref{sec:functional-requirements} are covered by defined use cases.
Table \ref{table:uc-x-fr} shows how all identified requirements are covered by use cases.

\begin{table}[h]
    \centering
    \begin{tabular}{|>{\columncolor{bananamania}}l | c | c | c | c | c | c | c | c | c |} 
        \hline
        \rowcolor{bananamania}
        \makecell[t]{functional requirement\\\hline{}use case} & \rot{FR1} & \rot{FR2} & \rot{FR3} & \rot{FR4} & \rot{FR5} & \rot{FR6} & \rot{FR7} & \rot{FR8} & \rot{FR9} \\
        \hline
        UC1 & $\bullet$ & $\bullet$ & \, & \, & \, & \, & \, & \, & \, \\
        \hline
        UC2 & \, & \, & \, & \, & \, & \, & \, & \, & $\bullet$ \\
        \hline
        UC3 & \, & \, & $\bullet$ & $\bullet$ & \, & $\bullet$ & \, & \, & \, \\
        \hline
        UC4 & \, & \, & \, & \, & $\bullet$ & \, & $\bullet$ & \, & \, \\
        \hline
        UC5 & \, & \, & \, & \, & \, & \, & \, & $\bullet$ & \, \\
        \hline
   \end{tabular}
   \caption{Coverage of functional requirements by use cases}
   \label{table:uc-x-fr}
\end{table}

Figure \ref{fig:use-case-diagram} shows the use cases diagram together with relationships between them.
Since questionnaire migration feature is integrated into the system only for authenticated users, the diagram contains exactly one actor.
The analysis of non-functional requirements shows that all authenticated users (regardless of role) are allowed to manage migrations.
Therefore only one role is used.

\image[0.8\textwidth]{fig:use-case-diagram}{Use cases relationship diagram}{use-case-diagram.pdf}
