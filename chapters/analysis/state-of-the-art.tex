\section{State-of-the-art}

As has been already stated, the \gls{dsw} application is being developed in cooperation of ELIXIR CZ and ELIXIR NL research groups.
It is currently split into multiple independent parts.

The whole system is available as an open-source\footnote{Application repositories are available at GitHub under the organization page at https://github.com/ds-wizard.}.
That said, the source code, documentation and the development process are publicly accessible and open to contribution.

The most significant part of the project is the server application called DSW-Server (in \gls{dsw} terminology called \textit{portal}).
This application implements all the functionality and business logic of the system.
Using standard terminology we can reffer to it as a backend.

Since the portal application does not provide any \gls{ui} (it only exposes \glsentryshort{api} using \glsentryshort{rest} standard), there is an complementary web application called DSW-Client (or, in \gls{dsw} terminology, called just \textit{client} for short) which does just that.
Using standard terminology such application is reffered to as a frontend.

\todo{Make sure the server part description and development state are both correct}

\subsection{DSW-Server}

The backend part of the application is written using Haskell programming language.
The main goal of this portal is to create data management plans. \reword{Most of the sentences starts with the word "the", reword it so it does not feel as machine generated.}
To create a data plan, the system has to contain at least one knowledge model.
At the time of writinng, \gls{dsw} is shiped with prebuild knowledge model called \textit{core}.

Knowledge models are represented using \gls{json} and stored in a file on server file system.
Data stewards then use these models to create a questionaries.
The role of researchers is to fill in the questionaries with meta informations about their research.

Filled-in questionaries are used by data stewards to create management plans.
This flow basically covers system features.

We however have silent assumptions in the flow described above.
The portal also have to support following features in addition to the mentioned business logic:

\begin{itemize}
    \item User management (roles, authorization and authentication),
    \item Package administration (cration, editing and versioning),
    \item Data persistency (database connection, file system integration)
    \item \glsentryshort{rest} \glsentryshort{api}.
\end{itemize}

All of the mentioned functions are split into independent Haskell modules.
The \glsentryshort{rest} \glsentryshort{api} layer is built using Scotty framework\footnote{Scotty is an open-source web framework written in haskell, inspired by Ruby's Sinatra. It is an lightweight alternative to frameworks like Yesod or Spock. The source code is available on GitHub: https://github.com/scotty-web/scotty.}.

\subsection{DSW-Client}

\subsection{System architecture}

\subsection{Knowledge model migrations}
