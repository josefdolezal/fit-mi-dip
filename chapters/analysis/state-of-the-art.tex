\section{State-of-the-art}

As has been already stated, the DSW application is being developed in cooperation of ELIXIR CZ and ELIXIR NL research groups.
The development is currently split into multiple independent parts.

The whole system is available as an open-source\footnote{Application repositories are available at GitHub under the organization page at https://github.com/ds-wizard.}.
That said, the source code, documentation, and the development process are publicly accessible and open to contribution.

The most significant part of the project is the server application called DSW-Server (in \gls{dsw} terminology called \textit{portal}).
This application implements all the functionality and business logic of the system.
Using standard terminology, we can refer to it as a backend.

Since the portal application does not provide any \gls{ui} (it only exposes \glsentryshort{api} using \glsentryshort{rest} standard), there is an complementary web application called DSW-Client (or, in \gls{dsw} terminology, called just \textit{client} for short) which does just that.
Using standard terminology, such application is refered to as a frontend.

\todo{Make sure the server part description and development state are both correct}

\subsection{DSW-Server}

The backend part of the application is written using the Haskell programming language.
The primary goal of this portal is to create data management plans. \reword{Most of the sentences starts with the word "the", reword it so it does not feel as machine generated.}
At least one knowledge model must be added to the system before data plans can be generated.
At the time of writing, \gls{dsw} is shipped with a prebuild knowledge model called \textit{core}.

Knowledge models are represented using \gls{json} and stored in a file on the server file system.
Data stewards then use these models to create questionaries.
The role of researchers is to fill in the questionaries with meta information about their research.

Filled-in questionaries are used by data stewards to create management plans.
This flow mostly covers system features.

We, however, have silent assumptions in the flow described above.
The portal also has to support the following features in addition to the mentioned business logic:

\begin{itemize}
    \item User management (roles, authorization and authentication),
    \item Package administration (cration, editing and versioning),
    \item Data persistency (database connection, file system integration)
    \item \glsentryshort{rest} \glsentryshort{api}.
\end{itemize}

All of the mentioned functions are split into independent Haskell modules.
The \glsentryshort{rest} \glsentryshort{api} layer is built using Scotty framework\footnote{Scotty is an open-source web framework written in Haskell, inspired by Ruby's Sinatra. It is a lightweight alternative to frameworks like Yesod or Spock. The source code is available on GitHub: https://github.com/scotty-web/scotty.}.

\subsection{DSW-Client}

DSW-Client is a frontend application written in Elm programming language\footnote{Elm is a type-safe programming language used for web applications frontend. The source code written in Elm is transpiled into JavaScript and interpreted using a web browser. Elm website: https://elm-lang.org.}.
It provides a web-based user interface for the server application.

The implementation was done by Ing. Jan Slifka as a part of his master's thesis in 2018.
\todo{Validate the date of frontend application}

The application is built using Elm Architecture.
The logic of such application is split into three separated parts:

\begin{description}
    \item[Model] -- represents applicaton state,
    \item[Update] -- allows to update the state,
    \item[View] -- interprets state using HTML.
\end{description}

This architecture has a similar approach to state modification as a popular state container called Redux.
Redux is also used for frontend applications, most commonly with React view library.

In oppose to React and Redux, Elm language was built with an architecture pattern in mind from the beginning.
According to the official website\cite{elm-speed}, Elm is up to two times faster than same application written in React.

The application is fully rendered locally in the user's web browser.
That makes the server and client applications fully independent of each other.
Development of both parts is, therefore, break up into two separate repositories.

\subsection{System architecture}

\subsection{Knowledge model migrations}

\subsection{Deployment}
