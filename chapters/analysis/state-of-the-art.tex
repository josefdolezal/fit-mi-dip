\section{State-of-the-art}

As has been already stated, the \gls{dsw} application is being developed in cooperation of ELIXIR CZ and ELIXIR NL research groups.
It is currently split into multiple independent parts.

The whole system is available as an open-source\footnote{Application repositories are available at GitHub under the organization page at https://github.com/ds-wizard.}.
That said, the source code, documentation and the development process are publicly accessible and open to contribution.

The most significant part of the project is the server application called DSW-Server (in \gls{dsw} terminology called \textit{portal}).
This application implements all the functionality and business logic of the system.
Using standard terminology we can reffer to it as a backend.

Since the portal application does not provide any \gls{ui} (it only exposes \glsentryshort{api} using \glsentryshort{rest} standard), there is an complementary web application called DSW-Client (or, in \gls{dsw} terminology, called just \textit{client} for short) which does just that.
Using standard terminology such application is reffered to as a frontend.

\todo{Make sure the server part description and development state are both correct}

\section{DSW-Server}

The backend part of the application is written using Haskell programming language.
The main goal of the portal is to create data management plans.
Those are made using completed questionaries.
Questionaries are made by data stewards on top of existing knowledge models.

\reword{Most of the sentences starts with the word "the", reword it so it does not feel as machine generated.}

Standard flow is then as follows:

\begin{enumerate}
    \item Data steward creates an questionnaire using knowledge model or its localization (so called \textit{package}),
    \item researcher fills up the questionnaire,
    \item portal generates the data management plan based on the completed questionnaire.
\end{enumerate}

This simple flow

\section{DSW-Client}

\section{System architecture}

\section{Knowledge model migrations}
