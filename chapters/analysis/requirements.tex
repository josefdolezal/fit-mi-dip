\section{Requirements}\label{sec:requirements}

In this section, I will define functional and non-functional requirements for the \gls{dsw} portal.

\subsection{Functional requirements}\label{sec:functional-requirements}

This section will introduce a list of identified functional requirements to the reader.
There are the following requirements:

\requirement{FR1 -- User is notified about available upgrade}{%
User is visually notified when a newer version of the knowledge model is available for the given questionnaire.
}

\requirement{FR2 -- Create questionnaire migration}{%
This requirement enables the user to create new questionnaire migration from current knowledge model to its newer version.
}

\requirement{FR3 -- Explore questionnaire on new knowledge model}{%
User can see a preview of the migrated questionnaire before the questionnaire is fully migrated.
This includes a preview of all questionnaire nodes together with the user's replies from previous knowledge model version.
}

\requirement{FR4 -- Guide user through the list of changes}{%
User is guided by the system through changes relevant to the migrated questionnaire.
}

\requirement{FR5 -- Resolve question change}{%
Question change may be marked as resolved, so the user does not have to remember which changes need to be reviewed before the questionnaire is migrated.
}

\requirement{FR6 -- Display node difference between versions}{%
Questionnaire nodes are shown in both the old version and new version with a visual representation of textual changes (removed, changed or added characters).
}

\requirement{FR7 -- Mark question change for review}{%
Allows user to mark questions which need to be carefully reviewed once the migration is done.
This enables the user to adjust open-ended answers or choose a different choose different answer in the hierarchical tree.
}

\requirement{FR8 -- Migrate questionnaire}{%
User is allowed to finish migration and upgrade questionnaire knowledge model version.
}

\requirement{FR9 -- Cancel questionnaire migration}{%
Questionnaire migration which was not finalized by the user may be canceled without affecting the original questionnaire.
}

\subsection{Non-functional requirements}\label{sec:non-functional-requirements}

Now, I would like to discuss the non-functional requirements of the system.
These requirements specify entitlements for system and further define the functional requirements.

\requirement{NR1 -- User authentication}{%
User is only allowed to create, modify, finish and cancel migration of questionnaires belonging to him.
The system administrator is allowed to migrate an arbitrary questionnaire.
Unauthenticated users are not authorized to access the migration feature at all.
}

\requirement{NR2 -- Migrated question consistency}{%
The question may be either unmarked, marked as resolved or marked for later review.
Other states are illegal.
}

\requirement{NR3 -- Preserve existing replies}{%
If the user's reply to a question is compatible with the new knowledge model (for example, the question type did not change), it must be preserved.
}

\requirement{NR4 -- State synchronization}{%
Question state is synchronized automatically without user interaction.
}

\requirement{NR5 -- Migration interoperability}{%
Questionnaire migrations are integrated into the existing system and are compatible with its domain model.
}
