\section{Requirements}\label{sec:requirements}

In this section, I will define functional and non-functional requirements for the \gls{dsw} portal.

\subsection{Functional requirements}\label{sec:functional-requirements}

This section will introduce list of identified functional requirements to the reader.
The requirements are following:

\requirement{FR1 -- User is notified about available upgrade}{%
User is visually notified when newer version of knowledge model is available for given questionnaire.
}

\requirement{FR2 -- Create questionnaire migration}{%
This requirement enables user to create new questionnaire migration from current knowledge model to its newer version.
}

\requirement{FR3 -- Explore questionnaire on new knowledge model}{%
User is able to see preview of migrated questionnaire before the questionnaire is fully migrated.
This includes preview of all questionnaire nodes together with answer from previous knowledge model version.
}

\requirement{FR4 -- Guide user through list of changes}{%
User is guided by system through changes relevant to migrated questionnaire.
}

\requirement{FR5 -- Resolve question change}{%
Question change may be marked as resolved so user does not have to remember which changes needs to be reviewed before the questionnaire is migrated.
}

\requirement{FR6 -- Display node difference betweeen versions}{%
Questionnaire nodes are shown in both old version and new version with visual representation of textual changes (removed, changed or added characters).
}

\requirement{FR7 -- Mark question change for review}{%
Allows user to mark questions which needs to be carefully reviewed once the migration is done.
This enables user to adjust open-ended answers or choose different choose different answer in hierarchical tree.
}

\requirement{FR8 -- Migrate questionnaire}{%
User is allowed to finish migration and upgrade questionnaire knowledge model version.
}

\requirement{FR9 -- Cancel questionnaire migraiton}{%
Questionnaire migration which was not finished by user may be canceled without affecting current questionnaire.
}

\subsection{Non-functional requirements}\label{sec:non-functional-requirements}

Now, I would like to discuss non-functional requirements of the system.
These requirements specify entitlements for system and further define functional requirements.

\requirement{NR1 -- User authentication}{%
User is only allowed to create, modify, finish and cancel migration of questionnaires belonging to him.
System administrator is allowed to migrate arbitrary questionnaire.
Unauthicated users are not authorized to access migration feature at all.
}

\requirement{NR2 -- Migrated question consistency}{%
Question may be either unmarked, marked as resolved or marked for later review.
Other states are ilegeal.
}

\requirement{NR3 -- Preserve existing replies}{%
If user's reply to question with new knowledge model (for example question type did not change), it must be preserved.
}

\requirement{NR4 -- State synchronization}{%
Question state is synchronized automatically without user interaction.
}

\requirement{NR5 -- Migration interoperability}{%
Questionnaire migrations are integrated into existing system and is compatible with its domain model.
}
