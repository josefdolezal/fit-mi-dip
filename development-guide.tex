\chapter{Development Guide}

This guide contains instructions for running both the server and client side applications in the development environment.

\section{Client-side Application}

The application's source code is available at GitHub in the following repository:

\begin{center}
    \url{git@github.com:ds-wizard/dsw-client.git}
\end{center}

For building the application, the \texttt{Node.js} environment together with the \texttt{yarn} package manager is needed.
The installation guide for the package manager is available at:

\begin{center}
    \url{https://yarnpkg.com/en/docs/install}
\end{center}

In the terminal, clone the repository change your working directory to the project root:

\begin{minted}[linenos, bgcolor=bgcode]{sh}
$ git clone git@github.com:ds-wizard/dsw-client.git
$ cd dsw-client
\end{minted}

Install the project dependencies:

\begin{minted}[linenos, bgcolor=bgcode]{sh}
$ yarn install
\end{minted}

Run the application:

\begin{minted}[linenos, bgcolor=bgcode]{sh}
$ yarn start
\end{minted}


The application is now running at \url{http://localhost:8080}.
It expects the backend to be available at \url{http://localhost:3000}.

\section{Server-side application}

For the server-side application, the Stack tool is required.
Installation guide for the Stack is available at:

\begin{center}
    \url{https://docs.haskellstack.org/en/stable/install_and_upgrade/}
\end{center}

You will also need a MongoDB database and a RabbitMQ messaging server.
For simplicity, there is a \texttt{Docker compose} configuration publicly available at GitHub:

\begin{center}\small
    \url{https://github.com/josefdolezal/dockerfiles/tree/master/dsw-local}
\end{center}

Once your development environment is ready, clone the application repository and change your working directory to the project root:

\begin{minted}[linenos, bgcolor=bgcode]{sh}
$ git clone git@github.com:ds-wizard/dsw-server.git
$ cd dsw-server
\end{minted}

Create your application configuration file and fill in your settings:

\begin{minted}[linenos, bgcolor=bgcode]{sh}
$ cp config/app-config.example.cfg config/app-config.cfg
$ vim config/app-config.cfg
\end{minted}

Build the application from sources:

\begin{minted}[linenos, bgcolor=bgcode]{sh}
$ stack build
\end{minted}

Run the server in a development configuration:

\begin{minted}[linenos, bgcolor=bgcode]{sh}
$ stack run dsw-server
\end{minted}

The application is now running at \url{http://localhost:3000}.
